\chapter{Domain and SSL Setup \\
\small{\textit{-- Nataly Jimenez, Nicole Valdiviezo, Lakshya Vegiraju}}
\index{Domain and SSL Setup} 
\index{Chapter!Domain and SSL Setup}
\label{Chapter::DomainSSL Setup}}

\section{Step 1: Domain Name Setup}

\subsection{Overview}
For this project, our group registered a custom domain name to host our Overleaf instance. We selected the domain \textbf{audioviz.me}, which was purchased through \textbf{Namecheap}. This domain will later be connected to our Overleaf container and GitHub repository for version-controlled LaTeX compilation and deployment.

\subsection{Procedure}
\begin{enumerate}
    \item \textbf{Domain Registration:}
    \begin{itemize}
        \item Visited \url{https://www.namecheap.com/} and searched for an available domain name.
        \item Registered the domain \textbf{audioviz.me} under our group’s Namecheap account.
        \item Verified ownership via the confirmation email sent by Namecheap.
    \end{itemize}

    \item \textbf{DNS Configuration:}
    \begin{itemize}
        \item Logged into the Namecheap dashboard.
        \item Navigated to \textit{Domain List → Manage → Advanced DNS}.
        \item Created an \textbf{A record} pointing to the IP address of our Overleaf server or cloud host (e.g., DigitalOcean, AWS, or local machine).
        \item Optionally, added a \textbf{CNAME record} for subdomains (e.g., \texttt{overleaf.audioviz.me}) pointing to the root domain.
        \item Saved changes and waited for DNS propagation (typically 10–30 minutes).
    \end{itemize}

    \item \textbf{Verification:}
    \begin{itemize}
        \item Used \texttt{nslookup audioviz.me} and \texttt{ping audioviz.me} to confirm DNS resolution.
        \item Accessed the domain in a browser to ensure it correctly routed to our host IP.
    \end{itemize}
\end{enumerate}

\subsection{Next Steps}
\begin{itemize}
    \item Configure Nginx or Apache as a reverse proxy for the Overleaf container.
    \item Ensure ports 80 (HTTP) and 443 (HTTPS) are open on the server.
\end{itemize}

\section{Step 2: SSL Configuration Research}

\subsection{Overview}
To secure our domain with HTTPS, we researched methods to add and renew SSL certificates for the Overleaf instance. The preferred approach uses \textbf{Let’s Encrypt}, a free Certificate Authority that automatically renews every 90 days.

\subsection{Installation and Renewal Options}
\begin{enumerate}
    \item \textbf{Install Certbot (Let’s Encrypt client):}
    \begin{verbatim}
    sudo apt update
    sudo apt install certbot python3-certbot-nginx
    \end{verbatim}

    \item \textbf{Obtain a Certificate:}
    \begin{verbatim}
    sudo certbot --nginx -d audioviz.me -d overleaf.audioviz.me
    \end{verbatim}
    This command:
    \begin{itemize}
        \item Automatically configures Nginx for SSL.
        \item Downloads and installs certificates for both the root and subdomain.
    \end{itemize}

    \item \textbf{Test Auto-Renewal:}
    \begin{verbatim}
    sudo certbot renew --dry-run
    \end{verbatim}

    \item \textbf{Alternative (Manual or Dockerized Setup):}
    \begin{itemize}
        \item If running Overleaf inside Docker, mount certificate volumes:
        \begin{verbatim}
        docker run -d \
            -v /etc/letsencrypt/live/audioviz.me:/etc/letsencrypt/live/audioviz.me \
            -p 80:80 -p 443:443 \
            overleaf/overleaf
        \end{verbatim}
        \item Configure the Overleaf container to use SSL via the Nginx proxy container.
    \end{itemize}
\end{enumerate}

\subsection{Next Steps}
\begin{itemize}
    \item Integrate SSL configuration into the Overleaf Docker Compose file.
    \item Set up a cron job or systemd timer for automatic certificate renewal.
    \item Update the documentation with screenshots of the SSL verification (using \texttt{https://audioviz.me}).
\end{itemize}

\subsection{Install Docker and Docker Compose on droplet}

\begin{minted}{bash}
# Update system
sudo apt update && sudo apt upgrade -y

# Install Docker
curl -fsSL https://get.docker.com -o get-docker.sh
sudo sh get-docker.sh

# Install Docker Compose
sudo apt install docker-compose -y

# Add current user to Docker group
sudo usermod -aG docker $USER

\end{minted}

\section{Summary}
After completing Steps 1 and 2, the group will have:
\begin{itemize}
    \item A working domain \textbf{(audioviz.me)} that points to the Overleaf host.
    \item A secure HTTPS connection via Let’s Encrypt.
    \item A foundation to connect Overleaf with GitHub and compile LaTeX projects under version control.
\end{itemize}



\section{Step 3: Set up Overleaf container}
\begin{minted}{bash}
mkdir -p ~/overleaf
cd ~/overleaf
\end{minted}

\begin{minted}{bash}
nano docker-compose.yml
\end{minted}

\begin{minted}{bash}
version: '3.8'

services:
  overleaf:
    image: sharelatex/sharelatex:latest
    restart: always                   # Restarts if container crashes or droplet reboots
    ports:
      - "80:80"                       # Exposes HTTP
      - "443:443"                     # Exposes HTTPS (for SSL)
    environment:
      - ROOT_URL=https://audioviz.me  # The domain Overleaf will use
    volumes:
      - ./data:/var/lib/sharelatex
    -./tmp:/var/lib/sharelatex/tmp
      - /var/run/docker.sock:/var/run/docker.sock
\end{minted}

\begin{minted}{bash}
docker-compose pull
docker-compose up -d
\end{minted}

\subsection{Configure Overleaf Container}
\begin{itemize}
  \item Set up a custom Overleaf container using the latest LaTeX image with all required packages preinstalled.
  \item Installed additional packages (\texttt{biblatex}, \texttt{tikz}, \texttt{geometry}, and \texttt{hyperref}) to support group document formatting.
  \item Verified compilation by building our main \texttt{.tex} project 
  \item Adjusted the Dockerfile to ensure consistent LaTeX builds across all group members.
  \item Noted configuration details and build success in our update log.
\end{itemize}

\section{Step 4 – Connect Overleaf to GitHub}
\begin{itemize}
  \item Created a private GitHub repository for our Overleaf project and initialized it with a \texttt{README.md}.
  \item Linked Overleaf to the repository via the built-in GitHub integration.
  \item Pushed our project to GitHub and confirmed that all files (\texttt{.tex}, images, and \texttt{.bib}) synced correctly.
  \item Tested syncing in both directions 
  \item Documented connection steps and repository link in our update history table.
\end{itemize}

\section{Step 5 – Compile from Command Line}
\begin{itemize}
  \item Cloned the GitHub repository locally using the command:
  \begin{verbatim}
  git clone https://github.com/yourusername/yourproject.git
  \end{verbatim}

  \item Navigated into the project directory:
  \begin{verbatim}
  cd yourproject
  \end{verbatim}

  \item Compiled the LaTeX document with the following commands to ensure reproducible builds:
  \begin{verbatim}
  latexmk -pdf main.tex
  \end{verbatim}
  or alternatively:
  \begin{verbatim}
  pdflatex main.tex
  bibtex main.aux
  pdflatex main.tex
  pdflatex main.tex
  \end{verbatim}

  \item Installed missing LaTeX dependencies as needed:
  \begin{verbatim}
  tlmgr install <package-name>
  \end{verbatim}

  \item Verified that the compiled PDF matched Overleaf’s online output and noted any differences.
\end{itemize}

\section{Step 6 – Add Versioning to the Project}
\begin{itemize}
  \item Added the version number and the short Git commit hash on the document title page.
  \item Confirmed version traceability by cross-checking GitHub tags with Overleaf project history.
\end{itemize}
